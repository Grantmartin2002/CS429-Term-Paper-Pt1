\documentclass[11pt]{article}

\usepackage{fullpage,times,hyperref,graphicx}

% The next line is the title of your paper. Update it.
\title{Defining a System} 
% The next line is the author information for your paper. Update it.
\author{Grant Martin glm2367}
% The next line is the timestamp for your paper. DO NOT CHANGE IT.
\date{\today}

% The next line controls the formatting of your references. DO NOT CHANGE IT.
\bibliographystyle{plain} 

\begin{document}
\maketitle

Throughout the various sciences and a multitude of perspectives, the term system is used quite liberally in various ways. However, there are certain traits of these definitions that overlap. Taking this overlap into consideration, a system can be defined as a typically hierarchical group of multiple parts that interact with each other to make a whole. Changes to these parts, therefore, change and evolve the system in its entirety. In this paper, I will cross reference various scientific papers' proposals of the definition of system to synthesize the previously stated definition. Next, I will provide my own proposal for a necessary feature of systems. Taking both the papers and my own thoughts into consideration, I will then list the criteria for a group to be considered a system. Finally, I will determine if the World Wide Web is considered a system based on this criteria.
\par Crossreferencing Ackoff's claim of a system's parts' dependence as well as Lampson's implication of dependence, we can conclude that a system must consist of interdependent parts. In Russell  L.  Ackoff's paper, ``Systems thinking and thinking systems,'' Ackoff states that a system ``cannot be divided  into  independent  parts  or subgroups  of parts.''\cite{ackoff:1994} In Butler W. Lampson's paper, ``Hints for Computer System Design,'' Lampson implies this previously mentioned trait of dependency, stating that, since an interface, which is a system, ``embodies assumptions which are shared by more than one part of a system, and sometimes by a great many parts, it is very desirable not to change the interface.''\cite{lampson:1983} 
\par Based on both Lampson's and Reason's implication of a system's reliance on its parts, we can conclude that a system's parts change and evolve the system as a whole. In Butler W. Lampson's paper, ``Hints for Computer System Design,'' Lampson states that it can be ``unclear about how one choice will limit … freedom to make other choices, or affect the size and performance of the entire system.''\cite{lampson:1983} Similarly, In James  Reason's paper, ``Human error: models and management'', Reason claims that a single ``strategic [decision]'' made by a member of a system has ``the potential   for   introducing   pathogens   into   the system.''\cite{reason:2000}
\par Based on Lampson's example, Parnas' suggestion, and Simon's outright claim of requirement of the hierarchical nature of complex systems, we can conclude that, if complex, which systems most often are, systems are hierarchical with potentially many layers of subsystems. In Lampson's paper, ``Hints for Computer System Design'', he provides such an example, stating that a computer system ``has much more internal structure, and hence many internal interfaces.'' in comparison to an algorithm\cite{lampson:1983} Furthermore, in David Lorge Parnas' paper, ``Software Aspects of Defense systems'', Parnas states that ``The system should be divided into modules using information-hiding (abstraction) before writing the program begins.''\cite{parnas:1985} In Herbert A. Simon's article, ``The Architecture of Complexity,'' he thoroughly depicts the hierarchical nature of complex systems, stating that a complex system consists of `` interrelated subsystems, each of the latter being, in turn, hierarchic in structure until we reach some lowest level of elementary subsystem.'' \cite{simon:1962}
\par As a man of science, I tend to follow the definitions that the relevant scientific literature suggests. Thus, I believe that, as synthesized in the previous paragraphs, a system can be defined as a typically hierarchical group of multiple parts that interact with each other to make a whole, and that changes to these parts, therefore, change and evolve the system in its entirety. However, one property of systems that I believe that scientific literature fails to mention in their proposed definitions is that some independence between parts is necessary for something to be considered as a system. For example, computers have a CPU, a GPU, a motherboard, a power supply, a case (typically), RAM, and a hard drive, each of which are distinguishable. Thus, computers meet this criteria of a system. However, I would not consider a puddle of water a system. Sure, a puddle of water consists of many water molecules (a group of parts) interacting with each other to make a whole. It also would be affected as a whole if one molecule were to be changed or removed (its volume would decrease). However, puddles of water do not have distinguishable parts. All water molecules are molecularly identical with no specialization whatsoever. Therefore, I think it would be more accurate to call it a body of water molecules rather than a system
\par Taking all factors into account, for a group to be considered a system, it must meet the following criteria:
\begin{itemize}
\item It must consist of multiple parts
\item Its parts must interact with each other
\item Its parts must be able to be grouped into one whole
\item Its parts must be dependent enough so that changes to an individual part will affect the system in its entirety
\item Its parts must be independent enough so that each part is distinguishable
\end{itemize}
\par Let’s determine if the World Wide Web, WWW for short, meets these criteria. 
\par One fundamental piece of the WWW is a web browser. Web browsers are what send and receive information to the WWW. As explained in Tim Berners-Lee’s, the creator of the WWW’s, turing award article,``Sir Tim Berners-Lee'', the WWW utilizes ``HTTP (hypertext transfer protocol)'' in order to fulfill these interactions between a web browser and the WWW.\cite{Haigh:2016} Thus, if one were to consider a web browser as part of the WWW, which I certainly would considering a web browser is required to communicate with the WWW, then this serves as an example of how the WWW consists of multiple parts that interact with each other, satisfying the first and second criteria for being a system. Additionally, As described in Tim Berners-Lee’s paper, ``Web Architecture from 50,000 feet'',the WWW consists of multiple parts, including (but not limited to) ``the HTTP space'' as a communication protocol, various ``Data Formats'' including ``HTML'' and ``XML'', and the ``web'' of information itself.\cite{Berners-Lee:1998} All of these parts, along with others, interact with each other in specific ways, allowing them to be grouped into the one whole that we call the WWW. Thus, both the first, second, and third criteria for a group to be considered a system is satisfied.
\par In Tim Berners-Lee’s paper, ``Information Management: A Proposal,'' Berners-Lee proposes that in order to prevent the structural restrictions a ``fixed hierarchical system'' provides, an information management system should const of a ``web of notes with links'' ``between them.''\cite{Berners-Lee:1990} This idea became the foundation of the WWW we have today. Nearly all web pages are accessed through a reference from another, especially via search engines. Every node is a dependency of all the nodes that reference it, meaning a change in one node will affect all of the nodes that depend on it, fundamentally altering the WWW as a whole. Thus, the WWW satisfies the fourth criteria of a system. 
Furthermore, Berners-Lee emphasizes the distinguishability of each part of the WWW in his paper, ``Web Architecture from 50,000 feet'', stating that the distinguishability that a ``URI'' provides is ``core to the universality'' of the WWW.\cite{Berners-Lee:1998} Every webpage has a unique URI, better known as today as a URL, thus making the parts of the WWW distinguishable and satisfying the fifth and final criteria of a system.

\par In conclusion, as shown in the prior paragraphs, the WWW satisfies all 5 criteria for a group to be considered a system. Thus, the WWW is a system.
\par
\par Word Count: 1021

















	





	







\bibliography{term-paper-template}
\end{document}